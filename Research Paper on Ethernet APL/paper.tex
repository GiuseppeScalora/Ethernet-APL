\documentclass[conference]{IEEEtran}
\IEEEoverridecommandlockouts
% The preceding line is only needed to identify funding in the first footnote. If that is unneeded, please comment it out.
\usepackage{cite}
\usepackage{amsmath,amssymb,amsfonts}
\usepackage{algorithmic}
\usepackage{graphicx}
\usepackage{textcomp}
\usepackage{xcolor}
\usepackage{amsmath}
\usepackage{amssymb}
\def\BibTeX{{\rm B\kern-.05em{\sc i\kern-.025em b}\kern-.08em
    T\kern-.1667em\lower.7ex\hbox{E}\kern-.125emX}}
\begin{document}

\title{IoT - Ethernet APL\\
}

\author{\IEEEauthorblockN{ Giuseppe Scalora}
\IEEEauthorblockA{\textit{Electronic Engineering} \\
\textit{Hochschule Hamm-lippstadt}\\
Lippstadt, Germany \\
giuseppe.scalora@stud.hshl.de}

}

\maketitle

\begin{abstract}
This paper will focus on the research about the communication protocol called Ethernet APL or Advanced Physical Layer. Describing how the communication occurs, what are the main instruments and hardware needed in order to make this type of communication possible. The scope of the paper will also cover the difference between the main Ethernet protocol and the APL with a comparison with the other OSI layers used in the communication protocols. Towards the end a couple of main applications will be included in order to understand the focus area of this topic alongside with the benefits and cons which this protocol can deliver.
 
\end{abstract}

\begin{IEEEkeywords}

\end{IEEEkeywords}

\section{Introduction}



\section{History of Ethernet}

\section{Motivation}


\section{OSI-Layer}

\section{Advanced physical layer}

\section{Main Applications}

\section{Conclusion}




\begin{thebibliography}{00}
\bibitem{b1} 
\bibitem{b2} 
\bibitem{b3} 
\bibitem{b4} 
\bibitem{b5} 
\bibitem{b6} 
\bibitem{b7} 
\bibitem{b8} 
\bibitem{b9} 
\bibitem{b10}
\end{thebibliography}

\end{document}
