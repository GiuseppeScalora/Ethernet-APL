\documentclass[conference]{IEEEtran}
\IEEEoverridecommandlockouts
% The preceding line is only needed to identify funding in the first footnote. If that is unneeded, please comment it out.
\usepackage{cite}
\usepackage{amsmath,amssymb,amsfonts}
\usepackage{algorithmic}
\usepackage{graphicx}
\usepackage{textcomp}
\usepackage{xcolor}
\usepackage{amsmath}
\usepackage{amssymb}
\def\BibTeX{{\rm B\kern-.05em{\sc i\kern-.025em b}\kern-.08em
    T\kern-.1667em\lower.7ex\hbox{E}\kern-.125emX}}
\begin{document}

\title{IoT - Ethernet APL\\
}

\author{\IEEEauthorblockN{ Giuseppe Scalora}
\IEEEauthorblockA{\textit{Electronic Engineering} \\
\textit{Hochschule Hamm-lippstadt}\\
Lippstadt, Germany \\
giuseppe.scalora@stud.hshl.de}

}

\maketitle

\begin{abstract}
This paper will focus on the research about the communication protocol called Ethernet APL or Advanced Physical Layer. Describing how the communication occurs, what are the main characteristics and properties needed in order to make this type of communication a reality. The scope of the paper will also cover the difference between the main Ethernet protocol and the APL with a comparison with the other OSI layers used in the communication protocols. Towards the end some important main applications will be discussed and presented in order to understand the focus area of this topic alongside with the benefits and cons which this protocol can deliver.
 
\end{abstract}

\begin{IEEEkeywords}
Ethernet, communication protocols, APL, OSI layers
\end{IEEEkeywords}

\section{Introduction}
When it comes to automation networks, many challenging requirements make their appearance in the process industries. Applications of such type require continuous and stable operations working 24/7, alongside with reliability and the proper managements of thousands of signals. Then, another important factor to consider, is the usage within hazardous environments, areas in which atmosphere can be flamable or explosive, which are common in process industries. Standard Ethernet, therefore, is not considered an intrinsically safe standard for hazardous environments and a new standard had to be developed. This new standard being a newly technological Advanced Physical Layer, which will enable a much improved safety in the already existing Ethernet standard. Ethernet APL is a new standard proposed on June, 2021. As of what said, it is important to understand how new this technology is in the field.


\section{Motivation}
Nowadays process plants are seeking for an efficient engineering, fast commissioning and reliable production. The only mean to allow this is the combination of an interoperable work between industy 4.0 and the digital transformation, these allow for a higher potential which would reinforce the variables mentioned just before. The main porblems to solve being accordingly, low bandwidth, speed and/or laborious protocol conversion among the communication protocols. Therefore Ethernet APL is there to solve fill these lackings and provide an efficient digitalization on field technologies. Field devices will then be provided with enhanced connectivity, with no longer present distance and cabling problems. An insight of a possibility of having data analysis and productivity improvements is shown in Figure 1 below.\cite{b1}
\begin{figure}[htbp]
    \centerline{\includegraphics[scale=1.0]{fig1.jpg}}
    \caption{Overview on communication step through Ethernet APL \cite{b1}}
    \label{comm}
\end{figure}

 Ethernet APL as the name itself suggests, is based upon the classical ISO OSI layer, more specifically it is another one of the already known physical layers, more of this in the next sections. 

\section{State of the art}
\subsection{History of the Ethernet}
It was May 22, 1973, at Xerox Palo Alto Research Center. Bob Metcalfe started drafting the description of how an Ethernet network system, he himself invented, should work. The main principle being the interconnectivity between computing workstations and the sending of informations between computers and laser printers. This was considered as one of the best inventions at PARC, being the first high-speed LAN technology of communication. The usage of Ethernet has been a revolution in the world of technologies. Allowing for a better and improved information sharing between applications, Ethernet has become the most used internet communication protocol used throughout all the main corporations and even in domestic private use. \cite{b2}
\subsection{OSI Layer}
In order to understand what the OSI layer are and how they are defined, it is important to learn how why is it needed. The Open Systems Interconnect (OSI) model is a conceptual framework which describes communication or networking between systems, properly divided into seven layers, as to each belong an own function. The division into layers is needed to help the realisation of a communication network, with well defined levels it is made easier to find problems and understand better what is going on in a specific area of the network. The vendors in this field will always make sure to specify in which exact layer their products belong. As to the classification, it will be discussed in the following subsections.
Since the main concern of this research is to investigate and discuss in details the Physical layer or also called OSI Layer 1, the listing will start from top to bottom. The picture below will better describe and show the layers.
\begin{figure}[htbp]
    \centerline{\includegraphics[scale=0.45]{OSI-Model.png}}
    \caption{OSI layers \cite{b3}}
    \label{osi}
\end{figure}

\subsubsection{Application Layer}
Starting from layer 7, the so called application layer, is the layer which gets the closest in touch with the end-user, for example by displaying and storing information which have to be sent through a display or a communication mean. The user interacts with the software and processes of synchronisation and exchange of data but it does not touch directly the application itself. Examples of this include Web Browsers such as Firefox or Chrome, which display web pages through HTTP communication, or TelNet. 
\subsubsection{Presentation Layer}
The presentation layer or layer 6 is where encapsulation, encryption and decryption of data happen. It creates a communication bridge between network and application via proper translation of data. Also called syntax layer because it makes sure that an independence between data representation and data translation is present. Example of this is the conversion from and to a data structure file of XML type.
\subsubsection{Session Layer}
The layer 5, as the name suggests already, creates and handles a session between two or more computers. It provides the capability of checkpointing, recovery, resuming and terminating a session, functions commonly used in internet protocols such as TCP. Its implementation is performed mostly by using remote procedure calls which allow the programmer to execute processes on a remote machine, triggered by the machine being in used.
\subsubsection{Transport Layer}
The transport layer manages the exchange and transmission of data sequences  with variable lengths between a source and a destination host. It thoroughly controls over the transmission making sure that the processes of segmentation and desegmentation occur properly. If some data has not been delivered correctly, it is the job of the transport layer to make sure that that portion of data gets recovered and then redelivered correctly. In our everyday´s life we can see the transport layer being applied to the UDP or TCP internet ports, entities which can allow or not allow the transmission of data through them. 
\subsubsection{Network Layer}
If the transport layer deals then with ports in UDP and TCP protocols, the network layer handles the IP addresses of a network. It pratically deals with the main router functionalities and communication between routers. If there is an ingoing or outgoing connection from a router to a server, the first has to make sure that the sending and receiving of packets gets executed and routed using the most feasible and efficient track/route among the millions available in the distributed system.
\subsubsection{Data Link}
The layer 2, processes the transmission and node to node communication. Working on top of the physical layer, can detect and correct errors which can occur in the latter. The data link layer is a bit of an exception compared to the others, as it possesses two sublayers the MAC or Media Access Control and the LLC, logical link control. 
The MAC is responsible to allow the access to the media, working in form of centralized or distributed control. Both handling the communication between nodes. The MAC specifically handles the frame synchronization which is responsible to determind the start and end of a bitstream transmission. 
The LLC, on the other hand, multiplexes the running protocols of the data link layer and determines the mechanisms used for addressing and controlling the communication exchange between sending and recipient machine.
\subsubsection{Physical Layer}
The physical layer is where the electrical conversion happens. The streams of raw data gets converted from digital to electrical or radio signals. Within the specifications of the layer, voltage levels, impedances, pin layouts, frequency, cable construction parameters and timing are almost always present. Technically the physical layer is nonetheless the physical representation of the system. Bluetooth, Ethernet and USB are standards examples of the Physical layer.
\cite{b4}\cite{b5}
\section{Ethernet Advanced physical layer}
The advanced physical layer Ethernet-APL brings standardized Ethernet technology and its advantages to the field of process plants. The leading standardization organizations are working together with major industry partners of process industries to specify one single physical layer that meets the requirements of process automation. Ethernet-APL is designed to support various topologies including redundancy concepts to ensure flexibility according to the needs of a process plant.
\subsection{Main characteristics}
10 Mbps communication over a single twisted pair cable for long reach as defined in IEEE 802.3cg-2019, short 10BASE-T1L

Explosion protection with intrinsic safety for hazardous zones as defined in IEC TS 60079-47 (2-WISE = 2-Wire Intrinsically Safe Ethernet)

APL Port Profiles for definition of multiple power levels between power source and power load and required cable characteristics (preferred: fieldbus cable type A) and connection technology (screw terminals, spring terminals, M12 connectors)
\section{Main Applications}

\section{Conclusion}




\begin{thebibliography}{00}
\bibitem{b1} 
\bibitem{b2} C. Spurgeon, Ethernet. Beijing: O'Reilly, 2009.
\bibitem{b3} https://www.educba.com/what-is-osi-model/
\bibitem{b4} https://www.networkworld.com/article/3239677/the-osi-model-explained-and-how-to-easily-remember-its-7-layers.html
\bibitem{b5} https://en.wikipedia.org/wiki/OSI_model
\bibitem{b6} 
\bibitem{b7} 
\bibitem{b8} 
\bibitem{b9} 
\bibitem{b10}
\end{thebibliography}

\end{document}
